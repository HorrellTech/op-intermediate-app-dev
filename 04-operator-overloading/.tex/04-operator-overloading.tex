% Beamer slide template prepared by Tom Clark <tom.clark@op.ac.nz>
% Otago Polytechnic
% Dec 2012

\documentclass[10pt]{beamer}
\usetheme{Dunedin}
\usepackage{graphicx}
\usepackage{fancyvrb}

\newcommand\codeHighlight[1]{\textcolor[rgb]{1,0,0}{\textbf{#1}}}

\title{Operator Overloading and Dunder Methods}

\author[IN608]{Intermediate Application Development}
\institute[Otago Polytechnic]{
  Otago Polytechnic \\
  Dunedin, New Zealand \\
  Kaiako: Tom Clark
}
\date{}
\begin{document}

%----------- titlepage ----------------------------------------------%
\begin{frame}[plain]
  \titlepage
\end{frame}

%----------- slide --------------------------------------------------%
\begin{frame}
  \frametitle{Let's add two integers}
  
  Suppose we want to add two integers, say 2 and 3. Addition of integers
  is a function that takes two arguments and returns their sum. We might 
  expect to write it this way:
  
  \vspace{5mm}
  \texttt{add(2,3)}

  \vspace{5mm}
  or, if we're feeling particularly object oriented,

  \vspace{5mm}
  \texttt{2.add(3)}

        
\end{frame}

%----------- slide --------------------------------------------------%
\begin{frame}
  \frametitle{Let's add two integers}
  
    But we don't do that. We generally we write it this way:

  \vspace{5mm}
  \texttt{ 2 + 3}

  \vspace{5mm}
  because somebody did it like that in 1360 and people liked it. The 
  ``+'' symbol in this expression is an \emph{operator}.
      
\end{frame}

%----------- slide --------------------------------------------------%
\begin{frame}
  \frametitle{Wow, that's handy}
  
  The ``+'' operator notation is so handy that we use it in other places

  \vspace{5mm}
  \texttt{> [3, 1, 4] + [22, 11]}
  
  \texttt{[3, 1, 4, 22, 11]}

  \vspace{5mm}
  It's not just ``+'' that gets this treatment.
  
  \vspace{5mm}
  \texttt{> 3 * 'cat'}
  
  \texttt{'catcatcat'}

  \vspace{5mm}
   What if we wanted to do this for our own classes?    
\end{frame}

%----------- slide --------------------------------------------------%
\begin{frame}
  \frametitle{Dunder Methods}
  
  Every object in Python inherits some ``dunder'' methods. They're
  also commonly called ``magic methods''. We've already seen two of these, 
  \texttt{\_\_init\_\_()} and  \texttt{\_\_str\_\_()}. There are also other such methods
  that we can choose to implement.
  
  \vspace{5mm}
  It's bad practice to define new dunder methods, and it won't generally do what we want 
  anyway. Also, it's possible to abuse dunder methods by implementing them in unexpected ways.
  Don't do that.
\end{frame}

%----------- slide --------------------------------------------------%
\begin{frame}[fragile]
  \frametitle{Operator overloading}
  Suppose we define a class and would like to be able to add two objects of that 
  class using ``+''. This would be an example of \emph{opertator overloading}.
  
  \begin{verbatim}
  class Bill:
      def __init__(self, items):
          self.line_items = items # a list of item objects
          
      def __add__(self, other):
          new_list = self.line_items + other.line_items
          return Bill(new_list)
  .
  .
  .        
  b1 = Bill(some_items)            
  b2 = Bill(some_other_items)
  b3 = b1 + b2
  \end{verbatim}
\end{frame}

%----------- slide --------------------------------------------------%
\begin{frame}
  \frametitle{Programming Activity}
  
  \begin{enumerate}
    \item Pull the course materials repo.
    \item Create a new branch, \texttt{04-practical} in your practicals repo.
    \item Add a subdirectory,  \texttt{04-practical} and copy \texttt{04-practical.ipynb} from the class materials into it.
    \item Open a shell, cd to this directory, and run \texttt{jupyter notebook} to open the notebook. Complete the first questions.
    \item We will discuss results in 30ish minutes.
  \end{enumerate}      
\end{frame}

%----------- slide --------------------------------------------------%
\begin{frame}
  \frametitle{Other Dunder Methods}
  
  There are other handy dunder methods to implement that aren't strictly 
  used for operator overloading 
  
\end{frame}

%----------- slide --------------------------------------------------%
\begin{frame}[fragile]
  \frametitle{\_\_str\_\_  and \_\_repr\_\_}
  
  We've already seen that we can use the \texttt{\_\_str\_\_()}
  method to control what happens when our object is passed as an argument to 
  \texttt{print()}. The general idea is that \texttt{\_\_str\_\_()} should return 
  a user-friendly string.
  
  \vspace{5mm}
  \texttt{\_\_repr\_\_()} is similar, but it should return a programmer-friendly string. A
  good idea is to return a string that matches a call to the constructor method that returns
  the given object.
  
  \begin{verbatim}
  class Cat:
      def __init__(self, name):
          self.name = name
          
      def __repr__(self):
          return f'Cat("{self.name}")'
          
  c  = Cat("Larry")
  repr(c)                 
  \end{verbatim}           
  
\end{frame}

\begin{frame}[fragile]
  \frametitle{\_\_len\_\_}
  
  We have seen that we can call \texttt{len()} on a list or
  string to get its length. If we want to make this work for our own classes,
  we implement \texttt{\_\_len\_\_()}.

  \begin{verbatim}
  class Bill:
      def __init__(self, items):
          self.line_items = items # a list of item objects
          
      def __len__(self):
          return len(self.line_items}
  \end{verbatim}        

\end{frame}

%----------- slide --------------------------------------------------%
\begin{frame}
  \frametitle{Conclusions}
  
   Operator overloading is somewhat controversial. Not all languages
   support it and some programmers think it's a bad idea. I find it 
   to be useful.
   
   Implementing appropriate dunder methods is a very ``Pythonic'' way to
   do things. Well implemented dunder methods can make your classes easy 
   and productive to use.
        
\end{frame}

\end{document}
